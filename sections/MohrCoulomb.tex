\section{Mohr-Coulomb-Bruchkriterium}
	\begin{minipage}{\linewidth}
	Verformung = $\Delta$V + $\Delta$Form \\
	$\rightarrow$ bis Bruch linear Elastisch (Hook: $\sigma = \varepsilon \cdot E_{oed}$), danach Mohr-Coulomb \\
	\qquad $\rightarrow$ $E_{oed} = M_E = \frac{4}{3} \cdot G + \kappa$ \\
	\qquad $\rightarrow$ $E_{normal} = \frac{9 \cdot \kappa \cdot G}{3 \cdot \kappa + G}$



\subsection{Volumetrische ($\Delta$V) Verformung}
	\begin{align*}
		p`				&= \frac{\sigma_1 + \sigma_2 + \sigma_3}{3}	 \\
		\varepsilon_v	&= \frac{\Delta V}{V} = \varepsilon_1 + 2 \cdot \varepsilon_3 \\
	\end{align*}
	
\subsection{Deviatorische ($\Delta$ Form aus Schub) Verformung}
	\begin{align*}
		q				&= \sqrt{\frac{(\sigma_1 - \sigma_2)^2 + (\sigma_1 - \sigma_3)^2 + (\sigma_2 - \sigma_3)^2}{2}} \\
		\varepsilon_s	&= \frac{2}{3} \sqrt{\varepsilon_1^2 + \varepsilon_2^2 + \varepsilon_3^2 - (\varepsilon_1 \cdot \varepsilon_2 + \varepsilon_2 \cdot \varepsilon_3 + \varepsilon_1 \cdot \varepsilon_3)} \\
	\end{align*}
	
	
	
\subsection{Oedometer}
	\textbf{Visualisierung}
	\begin{tabular}{ll}
		$\Delta$Volumen		& $p= \frac{\sigma_1 + 2 \cdot \sigma_3}{3}$ \\
							& $p`= p - u$ \\
							& $\varepsilon_v= \frac{\varepsilon_1}{3}$ \\
		Deviator ($\Delta$Form) &$q`= q = \sigma_1 - \sigma_3$ \\
							& $\varepsilon_s= \frac{2}{3} \cdot \varepsilon_1$ \\
	\end{tabular}
	\end{minipage}


\begin{minipage}{\linewidth}
\subsection{Ebenerspannungsraum}

\textbf{Tau-Sp-Diagramm}
	\begin{align*}
		\tau 		&= c`+ \sigma \cdot tan(\varphi) \\
		\tau		&= b + s`\cdot tan(\beta) \\
		\rightarrow c&= \frac{b`}{cos(\varphi)} \\
		\rightarrow sin(\varphi) &= tan(\beta) \\
	\end{align*}

\subsection{Dreidimensional}

\textbf{p-q-Diagramm} \\
	\begin{align*}
		q			&= \sigma = M \cdot p + d \\
		M			&= \frac{3 \cdot sin(\varphi)}{\sqrt{3} \cdot cos(\Theta) - sin(\Theta) \cdot sin(\varphi)} \rightarrow \Theta = Theta =Winkel der Sp. welche Probe belastet\\
	\end{align*}
	
	\subsubsection{Triaxversuch AK}
	\begin{align*}
		M			&= \frac{6 - sin(\varphi)}{3 - sin(\varphi)} \\
		d			&= \frac{6 \cdot c`\cdot cos(\varphi)}{3 - sin(\varphi)} \\
		\Theta		&= 30°
	\end{align*}
\end{minipage}
